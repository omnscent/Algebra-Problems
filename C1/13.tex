\subsection{体、域的基本概念}
\begin{prob}
设$R$是非零交换幺环。证明$R$是单环当且仅当$R$是域。
\end{prob}

\begin{prob}
设$K$是域,$\varphi :K\rightarrow R$是环同态。证明$\varphi \left(K\right)=\left\{0 \right\}$或$\varphi $是单射。
\end{prob}

\begin{prob}
证明有限整环是域。
\end{prob}

\begin{prob}
在$p$--进制域$\mathbb{Q}_p$中证明$\displaystyle \sum _{i=1}^{\infty }p^{i}=\frac{1}{1-p}$
\end{prob}

\begin{prob}
证明$p$--进整数环$\mathbb{Z}_p$的可逆元素乘法群
\begin{equation*}
\mathbb{Z}_{p}^{\times }=\left\{a\in \mathbb{Z}_{p}\vert \left\lvert a \right\rvert _{p}=1 \right\}.
\end{equation*}
\end{prob}

\begin{prob}
证明$p$--进整数环$\mathbb{Z}_p$的任一非零理想皆形如
\begin{equation*}
\left\{a\in \mathbb{Z}_{p}\vert v_{p}\left(a\right)\geqslant n \right\},
\end{equation*}
其中$n$为非负整数。
\end{prob}

\begin{prob}
证明$p$--进整数环$\mathbb{Z}_p$的任一非零理想皆形如$\mathfrak{M}^{n}$,其中$\mathfrak{M}$为$\mathbb{Q}_p$的赋值理想,$n$为非负整数。
\end{prob}

\begin{prob}
证明$p$--进整数环$\mathbb{Z}_{p}$等于可逆元素乘法群与赋值理想$\mathfrak{M}$的无交并。
\end{prob}

\begin{prob}
证明$\mathbb{H}_0=\left\{a+bI+cJ+dK\in \mathbb{H}\vert a,b,c,d\in \mathbb{Q} \right\}$是四元数体$\mathbb{H}$的子体。
\end{prob}

\begin{prob}
求四元数体$\mathbb{H}$的中心(即$\mathbb{H}$中与所有元素乘法可交换的全体元素)。
\end{prob}

\begin{prob}
证明四元数体的单位元素$\left\{\pm 1,\pm I,\pm J,\pm K \right\}$在乘法下组成一个群,叫做{\heiti 四元数群},记作$Q_{8}$。证明$Q_{8}$的每个子群都是正规子群。
\end{prob}

\begin{prob}
证明四元数群$Q_{8}$与$4$阶循环群的直和存在非正规子群。
\end{prob}

\begin{prob}
设$L$是含有两个以上元素的环。如果对于每个非零元素$a\in L$,都存在唯一的元素$b\in L$使得$aba=a$,证明:
\begin{enumerate}[$(1)$]
\item $L$无非零零因子;
\item $bab=b$;
\item $L$有1;
\item $L$是体。
\end{enumerate}
\end{prob}

\begin{prob}
设$L$是体,$a,b\in L,ab\neq 0,1$。证明{\heiti 华罗庚等式}:
\begin{equation*}
a-\left(a^{-1}+\left(b^{-1}-a\right)^{-1}\right)^{-1}=aba.
\end{equation*}
\end{prob}

\begin{prob}
设$F$是特征$p>0$的域。证明
\begin{equation*}
\left(a+b\right)^{p}=a^{p}+b^{p},\ \forall a,b\in F.
\end{equation*}
\end{prob}

\begin{prob}
给出两个有限非交换群$G$,分别适合:
\begin{enumerate}[$(1)$]
\item $g^{3}=e,\forall g\in G$;
\item $g^{4}=e,\forall g\in G$.
\end{enumerate}
\end{prob}
