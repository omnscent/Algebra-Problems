\subsection{几种特殊类型的群}
\begin{prob}
设$g$为群$G$中的$rs$阶元素,其中$r$与$s$互素。证明存在$a,b\in G$满足$g=ab,o\left(a\right)=e,o\left(b\right)=s$且$a,b$都是$g$的方幂。
\end{prob}
\begin{prob}
如果群$G$中的元素$g$的阶与正整数$k$互素,证明方程$x^{k}=g$在$\left\langle g \right\rangle $内恰有一个解。
\end{prob}
\begin{prob}
证明有理数加法群的任一有限生成子群都是循环群。
\end{prob}
\begin{prob}
设$G$是有限生成的交换群。如果$G$的每个生成元的阶都有限,证明$G$是有限群。
\end{prob}
\begin{probx}
证明有限生成群的指数有限的子群也是有限生成的。
\end{probx}
\begin{probx}
设$G$是群。对于任一正整数$k$,令$G^{k}=\left\{g^{k}\vert g\in G \right\}$。证明$G$是循环群的充分必要条件是$G$的任意一个子群都是$G^{k}$这样的集合。
\end{probx}
\begin{probx}
设$p$是素数,$n$是正整数,$G={\rm Z}_{p^{n}}$。试确定${\rm Aut}\left(G\right)$。
\end{probx}
\begin{prob}
写出互不同构的所有的$36$阶交换群。
\end{prob}
\begin{prob}
求${\rm Z}_3\oplus {\rm Z}_{9}\oplus {\rm Z}_{9}\oplus {\rm Z_{243}}$的$9$阶循环和非循环子群的个数。
\end{prob}
\begin{prob}
证明$S_{n}$可以由$n-1$个对换$\left(1\ 2\right),\left(1\ 3\right),\cdots ,\left(1\ n\right)$生成,也可以由$\left(1\ 2\right),\left(2\ 3\right),\cdots ,\left(n-1\ n\right)$生成。
\end{prob}
\begin{prob}
证明$S_{n}$可以由对换$\left(1\ 2\right)$和轮换$\left(1\ 2\ \cdots \ n\right)$生成。
\end{prob}
\begin{prob}
如果$n$是大于$2$的偶数,证明$A_{n}$可以由$\left(1\ 2\ 3\right),\left(1\ 2\ 4\right),\cdots \left(1\ 2\ n\right)$生成,也可以由$\left(1\ 2\ 3\right),\left(2\ 3\ 4\right),\cdots ,\left(n-2\ n-1\ n\right)$生成。
\end{prob}
\begin{prob}
证明:如果$n$是大于2的偶数,$A_{n}$可以由$\left(1\ 2\ 3\right)$和$\left(2\ 3\ \cdots \ n\right)$生成;如果$n$是大于2的奇数,则$A_{n}$可以由$\left(1\ 2\ 3\right)$和$\left(1\ 2\ \cdots \ n\right)$生成。
\end{prob}
\begin{prob}
设$\sigma =\left(1\ 2\ \cdots \ n\right)$,证明$\sigma $在$S_{n}$中的中心化子是$\left\langle \sigma  \right\rangle $,并证明$\sigma $在$S_{n}$中的共轭类含有$\left(n-1\right)!$个元素。
\end{prob}
\begin{prob}
设$n>2$,证明$Z\left(S_{n}\right)=\left\{(1) \right\}$。
\end{prob}
\begin{prob}
设$n\geqslant 5$,证明$S_{n}$中只有一个非平凡的真正规子群,即$A_{n}$。
\end{prob}
\begin{prob}
设$G$是群,$N\trianglelefteq G,N\cap G'=\left\{e \right\}$。证明$N\leqslant Z\left(G\right)$。
\end{prob}
\begin{prob}
证明可解群的子群和商群都是可解群。
\end{prob}
\begin{prob}
设$H,K$都是群$G$的正规子群,$G/H$与$G/K$都可解。证明$G/H\cap K$也可解。
\end{prob}
\begin{prob}
如果群$G$恰有两个自同构,证明$G$必为交换群。
\end{prob}
\begin{prob}
证明阶大于2的有限群至少有两个自同构。
\end{prob}
\begin{prob}
证明${\rm Aut}\left({\rm Z}_{2}\oplus {\rm Z}_{2}\right)\cong S_{3}$。
\end{prob}
