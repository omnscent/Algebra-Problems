\subsection{环的若干基本知识}
(除特殊声明外,此习题中的环都是指交换幺环。)
\begin{prob}
设$I,J$为环$R$的理想,$I,J$互素,证明$IJ=I\cap J$。
\end{prob}
\begin{prob}
设$I_{1},\cdots ,I_{n}$为环$R$的两两互素的理想,证明
\begin{equation*}
I_{1}\cdots I_{n}=I_{1}\cap \cdots \cap I_{n}.
\end{equation*}
\end{prob}
\begin{prob}
设$I,J,K$为环$R$的理想,$IJ\subseteq K$且$I$与$K$互素,证明$J\subseteq K$。
\end{prob}
\begin{prob}
设$I,J,K$为环$R$的理想,$I,J\supseteq K$且$I$与$J$互素,证明$IJ\supseteq K$。
\end{prob}
\begin{prob}
设$p$是一个素数,$n$是大于1的整数,$R=\mathbb{Z}/(p^n)$,证明:
\begin{enumerate}[$(1)$]
\item $R$的元素不是可逆元就是幂零元;
\item $R$只有一个素理想,记作$P$;
\item 商环$R/P$是域。
\end{enumerate}
\end{prob}
\begin{prob}
设$\mathfrak{M}$是$p$进整数环$\mathbb{Z}_p$的赋值理想,证明对于任意正整数$n$,有$\mathbb{Z}_p/\mathfrak{M}^{n}\cong \mathbb{Z}/(p^{n})$。
\end{prob}
\begin{prob}
设$\varphi :R\rightarrow R_{1}$是把1映成1的环同态。如果$Q$是$R_{1}$的素理想,证明$P=\varphi ^{-1}(Q)$是$R$的素理想。如果$Q$是$R_{1}$的极大理想,$\varphi ^{-1}\left(Q\right)$一定是$R$的极大理想吗?
\end{prob}
\begin{prob}
若环$R$的一个素理想$P$包含有限多个理想$I_{i}\left(1\leqslant i\leqslant n\right)$的交,证明$P$包含某个$I_{i}$。
\end{prob}
\begin{prob}
若环$R$的一个理想$I$含于有限多个素理想$P_{i}\left(1\leqslant i\leqslant n\right)$的并,证明$I$含于某个$P_{i}$。
\end{prob}
\begin{prob}
证明有限环的素理想都是极大理想。
\end{prob}
\begin{prob}
设$p$为素数,写出分式环$\mathbb{Z}_{(p)}$(表成$\mathbb{Q}$的子集)。
\end{prob}
\begin{prob}
设$P$为环$R$的素理想(于是$R$可以视为$R_{p}$的子环)。
\begin{enumerate}[$(1)$]
\item 对于$R$的任一理想$I$,证明$I\cdot R_{p}$是$R_{p}$的理想;
\item 对于$R$的任一素理想$Q$,证明$Q\cdot R_{p}$是$R_{p}$的素理想或平凡理想$(1)$。
\item 证明$P\cdot R_{p}$是$R_{p}$唯一的极大理想;
\item 证明$Q\longmapsto Q\cdot R_{p}$给出$R$含于$P$的素理想的集合到$R_{p}$的素理想的集合的双射。
\end{enumerate}
\end{prob}
