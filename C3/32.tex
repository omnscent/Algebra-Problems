\subsection{整环内的因子分解理论}
\begin{prob}
构造一个不满足因子链条件的整环。
\end{prob}
\begin{prob}
设$R$是满足因子链条件的整环,证明$R$是唯一分解整环当且仅当$R$中任意两个元素都有最大公因子。
\end{prob}
\begin{prob}
设$R$是唯一分解整环,$S$是$R$的一个乘法封闭子集,$0\notin S$。证明分式环$S^{-1}R$也是唯一分解整环。
\end{prob}
\begin{prob}
举例说明唯一分解整环的子环不一定是唯一分解整环。
\end{prob}
\begin{prob}
设$R$是唯一分解整环,$P$是$R$的素理想。举例说明商环$R/P$不一定是唯一分解整环。
\end{prob}
\begin{prob}
证明一元多项式环$\mathbb{Z}[x]$的任一主理想都不是极大理想。
\end{prob}
\begin{prob}
设$K$是域。系数在$K$中的形式幂级数$\displaystyle \sum_{i=0}^{\infty }a_{i}x^{i}(a_{i}\in K,x\text{为不定元})$的全体在通常的加法和乘法下构成一个环,称为$K$上的一元{\bfseries 形式幂级数环},记为$K[[x]]$。
\begin{enumerate}[$(1)$]
\item 设$\displaystyle f\left(x\right)=\sum_{i=0}^{\infty }a_{i}x^{i}\in K[[x]]$,证明$f\left(x\right)$是$K[[x]]$的可逆元当且仅当$a_{0}\neq 0$;
\item 证明$K[[x]]$是主理想整环。
\end{enumerate}
\end{prob}
\begin{prob}
证明:主理想整环中的非零素理想是极大理想。
\end{prob}
\begin{prob}
设$R$是主理想整环,$a,b,d\in R$,则$\left(a,b\right)=\left(d\right)$当且仅当$d$是$a,b$的最大公因子。
\end{prob}
\begin{prob}
设$R$是主理想整环,$D$是包含$R$的主理想整环,$a,b,d\in R$,$d$是$a,b$在$R$中的最大公因子。证明$d$也是$a,b$在$D$中的最大公因子。
\end{prob}
\begin{prob}
设$R$是主理想整环,$P$是$R$的一个非零素理想。证明在分式环$R_{P}$上可以定义一个绝对值函数,满足第一章$\S1.3$例$3.3$中的三个条件。
\end{prob}
\begin{prob}
设$K$是代数数域(见第1章$\S1.3$例$1.3$)。$K$的元素$\alpha $称为一个{\bfseries 代数整数},如果$\alpha $是一个首项系数为$1$的整系数多项式的零点。设$d$是无平方因子的整数,$K=\mathbb{Q}\left(\sqrt{d}\right)$。
\begin{enumerate}[$(1)$]
\item 如果$d\equiv 2,3\pmod{4}$,证明$K$中代数整数的集合等于
\begin{equation*}
\left\{a+b\sqrt{d}\vert a,b\in \mathbb{Z} \right\};
\end{equation*}
\item 如果$d\equiv 1\pmod{4}$,证明$K$中代数整数的集合等于
\begin{equation*}
\left\{\left. a+b\frac{1+\sqrt{d}}{2}\right\vert a,b\in \mathbb{Z} \right\}.
\end{equation*}
由此证明$K$中的代数整数的全体构成一个环,称为$K$的{\bfseries 代数整数环}。
\end{enumerate}
\end{prob}
\begin{prob}
证明$\mathbb{Q}\left(\sqrt{-3}\right)$的代数整数环是欧几里得环。
\end{prob}
\begin{prob}
证明$\mathbb{Q}\left(\sqrt{2}\right)$的代数整数环是欧几里得环。
\end{prob}
\begin{prob}
证明$\mathbb{Q}\left(\sqrt{5}\right)$的代数整数环是欧几里得环。
\end{prob}
\begin{prob}
证明环$\mathbb{Z}[i]$(其中$i=\sqrt{-1}$)的可逆元素乘法群为$\left\{\pm 1,\pm i \right\}$。
\end{prob}
\begin{prob}
设$p$为素数。如果$p\equiv 1\pmod{4}$,证明存在$a,b\in \mathbb{Z}$,使得
\begin{equation*}
p=a^{2}+b^{2}
\end{equation*}
\end{prob}
\begin{prob}
证明环$\mathbb{Z}[i]$的不可约元(在相伴意义下)有且仅有以下三种:
\begin{enumerate}[$(1)$]
\item $1+i$;
\item $a+bi$,其中$a,b\in \mathbb{Z}$满足$a^{2}+b^{2}\equiv 1\pmod{4}$为素数;
\item $p\equiv 3\pmod{4}$为素数。
\end{enumerate}
\end{prob}
\begin{prob}
设$K$是域,$K[x,y]$是$K$上的二元多项式环,
\begin{equation*}
R=K[x,y]/\left(x^{3}-y^{3}\right).
\end{equation*}
\begin{enumerate}[$(1)$]
\item 证明$R$是整环;
\item 令$P_{0}=\left(\bar{x},\bar{y}\right)\left(\subseteq R\right)$,证明$P_{0}$是$R$的极大理想,且
\begin{equation*}
\left(P_{0}\cdot R_{P_{0}}\right)/\left(P_{0}\cdot R_{P_{0}}\right)^{2}
\end{equation*}
作为$K$向量空间的维数为2;
\item 令$P_{1}=\left(\bar{x}-1,\bar{y}-1\right)\left(\subseteq R\right)$,证明$P_{1}$是$R$的极大理想,且$\left(P_{1}\cdot R_{P_{1}}\right)/\left(P_{1}\cdot R_{P_{1}}\right)^{2}$作为$K$向量空间的维数为1。
\end{enumerate}
\end{prob}
\begin{prob}
设$R$是唯一分解整环,$K$为$R$的分式域,$f\left(x\right)$是$R[x]$中的首项系数为1的多项式。如果$g\left(x\right)$是$f\left(x\right)$在$K[x]$中的首项系数为1的因子,证明$f\left(x\right)\in R\left[x\right]$。
\end{prob}
\stepcounter{problemname}
\begin{tcolorbox}[breakable,colback=shadecolor,colframe=framecolor,title=\textbf{问题\arabic{problemname}.}({\bfseries Eisenstein判别法})]
设$R$是唯一分解整环,$f\left(x\right)=a_{n}x^{n}+a_{n-1}x^{n-1}+\cdots +a_{0}\in R\left[x\right]$。如果存在$R$的不可约元$p$满足$p\nmid a_{n},p\vert a_{i}\left(\forall i<n\right),p^{2}\nmid a_{0}$,证明$f\left(x\right)$在$F\left[x\right]$中不可约($F$是$R$的分式域)。
\end{tcolorbox}\par
\begin{prob}
判断下列多项式在一元多项式环$\mathbb{Q}\left(i\right)\left[x\right]$中是否可约:
\begin{enumerate}[$\left(1\right)$]
\item $x^{p-1}+x^{p-2}+\cdots +1$,其中$p$为素数;
\item $x^{4}+\left(8+i\right)x^{3}+\left(3-4i\right)x+5$。
\end{enumerate}
\end{prob}
