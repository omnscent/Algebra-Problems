\usepackage[top=2.5cm,bottom=2.5cm,left=2.5cm,right=2.5cm]{geometry}
\usepackage{graphicx}
\usepackage{geometry}
\usepackage{tikz}
\usepackage{amssymb}
\usepackage{amsthm}
\usepackage{amsmath}
\usepackage{bm}
\usepackage{csquotes}
\usepackage{xcolor}
\usepackage{fancyhdr}
\usepackage{enumerate}
\usepackage{fontspec}
\usepackage{framed}
\usepackage{tcolorbox}
\newcounter{problemname}[subsection]
\usetikzlibrary{positioning}
\makeatletter
\@addtoreset{equation}{section}
\makeatother
\renewcommand{\theequation}{\arabic{section}.\arabic{equation}}
\newtheorem{thm}{定理}
\newtheorem{dy}{定义}
\newtheorem{ex}{例}
\newtheorem{yl}{引理}
\newtheorem*{prop}{性质}
\newtheorem*{mt}{命题}
\newtheorem*{tl}{推论}
\tcbuselibrary{breakable}
\catcode`\。=\active
\catcode`\,=\active
\catcode`\;=\active
\newcommand{。}{.\ }
\newcommand{;}{;}
\newcommand{,}{,\ }
\newenvironment*{zhu}{\begin{proof}[\indent\bf 注]}{\renewcommand{\qedsymbol}{}\end{proof}}
\newsavebox{\name}
\newenvironment{yinyong}[1]
{
\sbox\name{-- #1}
\begin{description}
    \item \begin{center}``
}
{''\end{center}
    \hspace*{\fill}\nolinebreak[3]
    \hspace*{\fill}
    \usebox{\name}
    \end{description}
}
\definecolor{shadecolor}{RGB}{255, 255, 204}
\definecolor{framecolor}{RGB}{250, 190, 0}
\newenvironment{prob}{\stepcounter{problemname} \begin{tcolorbox}[breakable,colback=shadecolor,colframe=framecolor,title=\textbf{问题\arabic{problemname}.}] }{\end{tcolorbox}\par}
\newenvironment{probx}{\stepcounter{problemname} \begin{tcolorbox}[breakable,colback=shadecolor,colframe=framecolor,title=\textbf{*问题\arabic{problemname}.}] }{\end{tcolorbox}\par}
%
%可以用这个简单的形式
% \newtheorem{prob}[problemname]{问题}
% \newtheorem{probx}[problemname]{*问题}
%
\newcommand*{\dif}{\mathop{}\!\mathrm{d}}
\newcommand{\gs}[1]{{\begin{equation*}\begin{array}{*{20}{c}}\displaystyle#1\end{array}\end{equation*}}}
\newcommand{\gsx}[1]{{\begin{equation}\begin{array}{*{20}{c}}\displaystyle#1\end{array}\end{equation}}}
\newcommand{\h}[1]{{\heiti #1}}
\newcommand{\z}[1]{{\rm #1}\mathop{}}
\newcommand{\jc}[1]{\bfseries{#1}}
\newcommand{\jdz}[1]{\left\lvert #1 \right\rvert}
\newcommand{\fs}[1]{\left\lVert #1 \right\rVert}
\newcommand{\dkh}[1]{ \left\{ #1 \right\} }
\newcommand{\zkh}[1]{ \left[ #1 \right] }
\newcommand{\kh}[1]{ \left( #1 \right) }
\newcommand{\jkh}[1]{ \left\langle #1\right\rangle }
\newcommand{\yh}[1]{“#1”}
\bibliographystyle{plain}
\endinput